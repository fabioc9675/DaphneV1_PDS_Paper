\section{Introduction}
\label{sec:introduction}

The Deep Underground Neutrino Experiment (DUNE) is an international experiment aimed to shed light on some of the mysteries at the forefront of particle physics and astrophysics via the precise measurement of neutrino oscillation parameters. The Far detector, located at SURF in South Dakota, consists of four gigantic modules with a total of \SI{40}{\kilo\tonne} of fiducial mass. Each module is a  liquid argon time-projection chamber (LArTPC) used for neutrino detection and signal reconstruction. \cite{Abi_2020,Abi_2020_2,Abi_2020_3}

In addition to the charge collection allowed by LArTPC technology, in DUNE is foreseen the implementation of a photon detection system (PDS), taking advantage of the scintillation properties of the Argon. For the two first DUNE Far detector Modules, the PDS technology is based on the ARAPUCA concept~\cite{Machado_2016}. The front-end system for the PDS first far detector is DAPHNE (Detector electronics for Acquiring PHotons from NEutrinos).

The current paper presents the system characteristics with a summary of the performance achieved in different labs. Finally,  we offer the plans for the PD readout implementation in ProtoDUNE-II.

%Table example
%\begin{table}[htbp]
%\centering
%\caption{\label{tab:i} We prefer that tables have the caption on top.}
%\smallskip
%\begin{tabular}{lrc}
%\hline
%x&y&x and y\\
%\hline
%a & b & a and b\\
%1 & 2 & 1 and 2\\
%$\alpha$ & $\beta$ & $\alpha$ and $\beta$\\
%\hline
%\end{tabular}
%\end{table}